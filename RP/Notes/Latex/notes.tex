\documentclass[a4paper]{article}

\usepackage[utf8]{inputenc}
\usepackage[T1]{fontenc}
\usepackage[slovene]{babel}

\usepackage{amsmath, amssymb}
\usepackage{listings}
\usepackage{lmodern}

\usepackage{hyperref}




% Create a command representation "\#1{}"
\newcommand{\command}[1]{
    \commandTwo{#1}{}\
}

% Create a command representation "\#1{#2}"
\newcommand{\commandTwo}[2]{
    \texttt{\textbackslash{}#1\{#2\}}\
}

% Create a command representation "\#1[#2]{#3}"
\newcommand{\commandThree}[3]{
    \texttt{\textbackslash{}#1[#2]\{#3\}}\
} 

\newcommand{\package}[1]{
    \commandTwo{usepackage}{#1}
}


\begin{document}

\title{LatexNotes}
\author{Matic Zavadlal}

\maketitle

\begin{abstract}
    We will learn how to use VSCode and how to write LaTeX.
\end{abstract}

\section{Moving documents using bash}

You can move documents and folders around using

\begin{lstlisting}[language=Bash]
    mv <source> <target>
\end{lstlisting}

That's how you move around documents and folders using bash! 

\section{Math formulas}

We type math formulas in \LaTeX{} using "\\\[" and "\\\]".

\section{Lists in LaTeX}

We create lists in \LaTeX using one of the list types and enumerating items inside the clause.

\begin{itemize}
    \item itemize - for a bullet list
    \item enumerate for an enumerated list and
    \item description for a descriptive list.
\end{itemize}

\section{Macros}

\LaTeX{} allows you to compose custom "functions" called macros.
Each macro is defined using \texttt{\textbackslash{}newcommand}, and continue by listing arguments and implementation.

\begin{itemize}
    \item Because \LaTeX{} isn't very good with data structure you can only make the first parameter optional.
    \item We reference parameters using \texttt{\#n} where \texttt{n} represents the order of the parameter you are referring to.
    \item We override a default argument using square brackets \newline
            \command{commandName}{override}{otherArgs}
  
\end{itemize}

\section{Bibliography in \LaTeX{}}

A short note on how to cite bibliography in \LaTeX{}.

\subsection{Packages}

\begin{itemize}
    \item \package{hyperref} for hyper links in the pdf file.
    \item \package{makeidx} don't know yet.
\end{itemize}

\subsection{Creating bibliography}

\section{Other notes}

Random notes that had nowhere to go.

\begin{itemize}
   \item We can create new line in latex using \command{newline}\ , or by creating an empty line between two paragraphs.
   \item We should use $\sim{}$ (\emph{$Prof. \sim{}A.\sim{}Bauer$}) when using names.
\end{itemize}

\end{document}