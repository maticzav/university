% Ta datoteka vsebuje osnove LaTeXa. Uporabite jo tako,
% da primerjate njeno vsebino z generirano datoteko PDF.

% Kogar zanima več LaTeXu, lahko obišče https://www.latex-project.org

% LaTeX sprejme eno ali več datotek .tex in naredi PDF.
% V datoteko .tex pišemo besedilo in ukaze.

% Komentarji se začnejo z znakom % in veljajo do konca vrstice.

% Dokument se začne z ukazom \documentclass.
% Ukaz \documentclass ima v [...] nastavitve in v {...} tip dokumenta.

\documentclass[a4paper]{article}

% LaTeX ima veliko Å¡tevilo tipov dokumentov, najboj pogosti so:
%   article - članek
%   amsart - članki za združenje American Mathematical Society
%   beamer - prosojnice za predavanje
%   book - knjiga
%   letter - pismo

% Nastavitve so odvisne od tipa dokumenta, nekatere običajne so:
% a4paper ... format strani naj bo A4 (namesto ameriški letter)
% 10pt, 11pt, 12pt ... velikost pisave

% Sledi PREAMBULA, kjer se še ne piše nobenega besedila, ampak
% samo ukaze in nastavitve.

% LaTeX ima veliko število zelo uporabnih paketov. Običajno najprej
% povemo, katere pakete bomo uporabljali. To naredimo z ukazom
% \usepackage[...]{...} kjer v [...] damo dodatne nastavitve za paket
% in v {...} ime paketa.

% Za slovenščino moramo nastaviti kodno tabelo datoteke .tex in
% kodno tabelo datoteke PDF, da se pravilno obravnava Å¡umnike.
% LaTeXu povemo tudi, da uporabljamo slovenščino (da pravilno prikaže datum ipd).

\usepackage[utf8]{inputenc}   % vhodna datoteka je kodirana v sistemu UTF-8

% Ostale možnosti za kodiranje so:
%\usepackage[cp1250]{inputenc}   % Kodirna tabela za Windows, zastarelo
%\usepackage[latin2]{inputenc}   % Kodirna tabela za Linux, zastarelo

% Ker smo nastavili vhodno datoteko .tex na UTF-8, moramo poskrbeti,
% da tudi urejevalnik, s katerim urejamo .tex, uporablja UTF-8.

\usepackage[T1]{fontenc}      % izhodna datoteka je korirana v sistemu T1
\usepackage{fontenc}

\usepackage[slovene]{babel}   % uporabljamo slovenščino

% Nastavimo lahko tudi razne pisave, glej http://www.tug.dk/FontCatalogue/
% Za slovenščino je dobro nastaviti novejšo inačico pisave lmodern

\usepackage{lmodern}            % Boljši fonti, s pravimi šumniki
% Ostale pogoste izbire pisav:
% \usepackage{times}            % Times New Roman
% \usepackage{palatino}         % Palatino
% \usepackage{concrete}         % Pisava, ki jo je uporabil Donald Knuth v knjigi "Concrete mathematics"

% S tem smo zaključili pakete in nastavitve za slovenščino. Ostali paketi
% in nastavitve so odvisni od tega, kaj bomo počeli. V tej datoteki bomo
% prikazali osnove in hiper-povezave.

\usepackage{hyperref}            % Paket za povezave znotraj dokumenta in hiper-povezave

% To je konec preambule
% Sledi vsbina dokumenta, ki se začne z ukazom \begin{document} in konča z \end{document}.

\begin{document}

% Nastavimo naslov in avtorja.

\title{Osnovno o {\LaTeX}u}
\author{Andrej Bauer}

% Lahko bi nastavili tudi datum, če ga ne, se bo izpisal današnji datum
% \date{10.\ 10.\ 2017}

% Ukaz, s katerim dejansko naredimo naslov
\maketitle

% LaTeX pozna t.i. okolja, ki se začnejo in končajo z ukazoma
% \begin{imeOkolja} ... \end{imeOkolja}.
% Okolij je zelo veliko, zaenkrat bomo uporabili samo nekatera osnovna.

\begin{abstract}
  Zdravo!
\end{abstract}

% Članek je hierarhično razdeljen na razdelke (\section), podrazdelke (\subsection),
% podpodrazdelke (\subsubsection) in podpodpodrazdelke (\paragraph). V večini primerov
% potrebujemo samo razdelke in podarazdelke.

\section{Uporaba razdelkov}

% Oblikovalci pogosto potrebujejo kako besedilo, da prikažejo design spletne strani
% ali dokumenta. Tradicionalno se v ta namen uporablja "Loerm ipsum ..." glej
% http://www.lipsum.com. Tudi mi ga bomo, kar je bolje, kot da bi citirali Cankarja.

Lorem ipsum dolor sit amet, consectetur adipiscing elit. Nulla semper, ligula eget
porttitor lobortis, lacus ante posuere magna, eget dapibus lacus odio sit amet urna.
Curabitur eget tincidunt lacus, a mollis lectus. Donec non est a eros gravida venenatis
nec quis elit. Sed vulputate, neque eget pellentesque egestas, eros diam porttitor est,
non lobortis velit turpis vitae velit.

\subsection{Uporaba podrazdelkov}

Donec bibendum tempor metus, vitae porttitor nibh sodales sed. Proin et est non magna
aliquet finibus. Vivamus volutpat risus vel purus suscipit auctor in eget metus. Praesent
eget nisl ante. Donec gravida, diam non aliquam rhoncus, elit enim pellentesque augue,
vitae vulputate orci orci non lorem. Duis vestibulum sapien eget orci venenatis, id
scelerisque risus mattis. Fusce dignissim lacus in nulla lobortis pretium. Aenean sit amet
imperdiet tellus.

\subsubsection{To je podpodrazdelek}

To je besedilo podrazdelka.

\paragraph{To je podpodpodrazdelek}

Velikokrat v svojem življenju sem storil krivico človeku, ki sem ga ljubil. Taka krivica
je kakor greh zoper svetega duha: ne na tem ne na onem svetu ni odpuščena. Neizbrisljiva
je, nepozabljiva. Časih počiva dolga leta, kakor da je bila ugasnila v srcu, izgubila se,
utopila v nemirnem življenju. Nenadoma, sredi vesele ure, ali ponoči, ko se prestrašen
vzdramiš iz hudih sanj, pade v dušo težak spomin, zaboli in zapeče s toliko silo, kakor da
je bil greh šele v tistem trenutku storjen. Vsak drug spomin je lahko zabrisati. Črn madež
je na srcu in ostane za vekomaj.

\subsection{In Å¡e kak podrazdelek}

Začul sem tihe korake na stopnicah. Prišla je mati; stopala je počasi in varno, v roki je
nesla skodelico kave. Zdaj se spominjam, da nikoli ni bila tako lepa kakor v tistem
trenutku. Skozi vrata je sijal poševen pramen opoldanskega sonca, naravnost materi v oči;
večje so bile in čistejše, vsa nebeška luč je odsevala iz njih, vsa nebeška blagost in
ljubezen. Ustnice so se smehljale kakor otroku, ki prinaša vesel dar.

Jaz pa sem se ozrl in sem rekel z zlobnim glasom: ">Pustite me na miru!~\dots Ne maram zdaj!"<

\section{Zaviti oklepaji in ukazi}

Z zavitimi oklepaji \texttt{\{} in \texttt{\}} v LaTeXu združimo kos besedila ali ukazov,
da se obravnavajo kot celota.

Ukazi ali \emph{makroji} se v LaTeXu pišejo
%
\begin{center}
  \verb|\imeUkaza{...}|
\end{center}
%
ali
%
\begin{center}
  \verb|\imeUkaza{...}{...}{...}|
\end{center}
%
če ukaz sprejme več argumentov. Uporabnik lahko definira svoje ukaze, a o tem kasneje.
Ukazi brez argumenta ``pojejo'' presledek, zato jih pišemo v zavite oklepaje:
%
\begin{itemize}
\item \LaTeX je dokumentni sistem.
\item {\LaTeX} je dokumentni sistem.
\end{itemize}


\section{Izbira in velikost pisave}

Praviloma pisave izbiramo premišljeno, ali pa izbor kar prepustimo LaTeXu.
Besedilo lahko \emph{poudarimo} ali zapišemo \textbf{krepko}, ali pa \emph{\textbf{oboje skupaj}}.
Kaj se zgodi, če \emph{uporabimo poudarjeno \emph{znotraj} poudarjenega besedila}?
Seveda \textbf{lahko lahko nastavimo tudi \textnormal{običajno} pisavo}.
Besedilo lahko tudi \underline{podčrtamo}, vendar tega ne priporočamo, \underline{ker} \underline{je grdo}.

% Sans-serifna pisava je pisava brez "repkov" (serifov), glej https://en.wikipedia.org/wiki/Sans-serif
Pišemo lahko tudi v \textsf{sans-serifni pisavi} ali pa \textsc{z malimi velikimi črkami},
denimo \textsc{Python}. Ležeča pisava \textsl{ni ista reč kot} \emph{poudarjena pisava}.
Včasih uporabimo tudi \texttt{pisavo fiksne širine}, v kateri so vsi znaki enako široki.


\subsection{Velikosti pisav}

% Ukazi s katerimi nastavimo velikost pisave, a tega raje ne počnite na roke
\begin{center}
{\Huge Zdravljica} \\
{\Huge Živé naj vsi naródi,} \\
{\huge ki hrepené dočakat dan,} \\
{\LARGE ko, koder sonce hodi,} \\
{\Large prepir iz svéta bo pregnan,} \\
{\large ko rojak} \\
{\normalsize prost bo vsak,} \\
{\footnotesize ne vrag, le sosed bo mejak!} \\
{\scriptsize Kdor ne skače ni Slovenc!} \\
{\tiny $\forall x \,.\, \lnot\mathrm{skače}(x) \Rightarrow \lnot\mathrm{slovenc}(x)$}
\end{center}

\subsection{Ligature in kerningi}

Pozorno poglejte naslednje primere in povejte, kaj se je zgodilo. Nato se naučite, kaj so
\emph{ligature} in \emph{kerningi}:
%
\begin{itemize}
\item afina preslikava
\item {a}{f}{i}{n}{a} preslikava
\item AVTO!
\item {A}{V}{T}{O}!
\end{itemize}


\section{Naštevanje}

Neoštevilčeno naštevanje:
%
\begin{itemize}

\item Phasellus dolor odio, rhoncus et turpis eu, pharetra maximus nibh. Nullam maximus
  orci sit amet enim sodales accumsan. Fusce consectetur diam placerat dictum viverra.
  Cras massa nisl, ultrices in aliquet nec, iaculis nec erat.

\item Mauris neque magna, tincidunt ut ante quis, pulvinar venenatis ex. Duis sollicitudin
  scelerisque mollis. Integer id eros ac metus pretium vehicula et nec odio. Integer
  rhoncus lacinia velit in auctor. Etiam id ipsum a eros posuere pharetra.
\end{itemize}

Oštevilčeno naštevanje:
%
\begin{enumerate}

\item Phasellus dolor odio, rhoncus et turpis eu, pharetra maximus nibh. Nullam maximus
  orci sit amet enim sodales accumsan. Fusce consectetur diam placerat dictum viverra.
  Cras massa nisl, ultrices in aliquet nec, iaculis nec erat.

\item Mauris neque magna, tincidunt ut ante quis, pulvinar venenatis ex. Duis sollicitudin
  scelerisque mollis. Integer id eros ac metus pretium vehicula et nec odio. Integer
  rhoncus lacinia velit in auctor. Etiam id ipsum a eros posuere pharetra.
\end{enumerate}

Vgnezdeno naštevanje:
%
\begin{enumerate}
\item Lorem ipsum dolor sit amet, consectetur adipiscing elit.
\item Phasellus dolor odio, rhoncus et turpis eu, pharetra maximus nibh:
  \begin{itemize}
  \item Nullam maximus orci sit amet enim sodales accumsan.
  \item Fusce consectetur diam placerat dictum viverra.
    \begin{itemize}
    \item Integer id eros ac metus pretium vehicula et nec odio.
    \item Integer rhoncus lacinia velit in auctor.
    \item Etiam id ipsum a eros posuere pharetra.
    \end{itemize}
  \item Cras massa nisl, ultrices in aliquet nec, iaculis nec erat:
    \begin{enumerate}
    \item Mauris neque magna, tincidunt ut ante quis, pulvinar venenatis ex. Duis
      sollicitudin scelerisque mollis.
    \item Integer id eros ac metus pretium vehicula et nec odio. Integer rhoncus lacinia
      velit in auctor.
    \item Etiam id ipsum a eros posuere pharetra.
    \end{enumerate}
  \end{itemize}
\end{enumerate}

\section{Posebni znaki in presledki}

\subsection{Posebni znaki}

Znaki
%
\begin{verbatim}
  # $ % & \ ^ _ { } ~
\end{verbatim}
%
imajo poseben pomen, saj se uporabljajo za komentarje, ukaze, matematični način itd. Če
jih želimo zapisati, to storimo takole:
%
\begin{center}
\begin{tabular}{cl}
znak & ukaz \\
\# & \verb|\#| \\
\$ & \verb|\$| \\
\% & \verb|\%| \\
\& & \verb|\&| \\
{\textbackslash} & \verb|{\textbackslash}| \\
{\textasciicircum} & \verb|{\textasciicircum}| \\
\_ & \verb|\_| \\
\{ & \verb|\{| \\
\} & \verb|\}| \\
{\textasciitilde} & \verb|{\textasciitilde}| \\
\end{tabular}
\end{center}
\subsection{Narekovaji}

Takole napišemo `enojne' narekovaje. Poiščite ta dva znaka na tipkovnici. Dvoje narekovaje
pišemo ``takole''. Pravzaprav s tem dobimo angleške narekovaje. Slovenski narekovaji so
">takšni"< ali pa "`takšni"'.

Napačni načini pisanja narekovajev: znaka za levi in desni narekovaj nista enaka, zato se
\emph{ne} piše 'takole' ali ''takole''. Še posebej pa se ne piše "takole".

\subsection{Vezaji, pomišljaji in minusi}
\label{sec:vezaji-pomilj-minusi}


\subsection{Presledki}

V {\LaTeX}u poznamo več vrst presledkov, zato se naučimo uporabljati. Navadne presledke
pišemo s presledki, pri čemer ima    več      zaporednih presledkov enak učinek kot en sam.

Poznamo še \emph{krepki} ali \emph{nedeljivi} presledek, ki se ga piše z znakom tilda
\texttt{\char126}. Ta presledek prepreči prelom vrstice in ga uporabimo, kadar bi bil
prelom videti grd. Primer je inicialka in ime, na primer N.~Tesla.

Presledek za piko {\LaTeX} obravnava drugače kot ostale presledke. Tak presledek je bolj
``raztegljiv'' in zato moramo vedno označiti presledke, ki sledijo piki, a niso konec
stavka. To naredimo tako, da pred presledek zapišemo poševnico \texttt{\char92} ali tildo
\texttt{\char126}. Tako pišemo N.~Tesla in ne N. Tesla. Včasih si ne moremo privoščiti
celega zaporedja nedeljivih presledkov, takrat uporabimo navaden deljivi presledek.
Denimo, če bi zapisali akad.~prof.~dr.~Franc Forstnerič, bi moral {\LaTeX} zaradi tega
deliti več besed, zato je boje pisati akad.\ prof.\ dr.~Franc Forstnerič.






\subsection{Odstavki}

Ena ali več praznih vrstic označuje nov odstavek. To pomeni, da ne smemo po nepotrebnem
delati praznih vrstic, kar se še posebej rado dogaja pred in za matematično formulo.

Pravilno pišemo
\[
   a^2 + b^2 = c^2,
\]
brez praznih vrstic. Če pa zapišemo prazno vrstico,
\[
   a^2 + b^2 = c^2,
\]

na primer za formulo, potem bo besedilo zamaknjeno, kot da se je začel nov odstavek,
čeprav se je nadaljeval stavek.



\section{Nekatera druga okolja}

Daljši navedek:
%
\begin{quote}
  No, potlej je tista pošast ali tisti peklenski škrat plezal gor ob barki prav na glas:
  škreb! škreb! škreb! Kakor je na vrh prišel, pa se ti je iz oči v oči meni nasproti
  postavil. Sveta mamka božja sedem križev in težav --- sem jaz dejal --- pa sem zavzdignil
  bridko sabljo pa sem zamahnil pa sem čez glavo ubral in loputnil: lop! --- pa sem ga
  presekal, samega hudiča sem presekal, na dva kosa!
\end{quote}

\noindent
Centrirano besedilo:
%
\begin{center}
Overhead the albatross hangs motionless upon the air\\
And deep beneath the rolling waves in labyrinths of coral caves\\
The echo of a distant tide\\
Comes willowing across the sand\\
And everything is green and submarine
\end{center}

\noindent
Poravnano levo:
%
\begin{flushleft}
And no one showed us to the land\\
And no one knows the wheres or whys\\
But something stirs and something tries\\
And starts to climb towards the light
\end{flushleft}

\noindent
Poravnano desno:
%
\begin{flushright}
Strangers passing in the street\\
By chance two separate glances meet\\
And I am you and what I see is me\\
And do I take you by the hand\\
And lead you through the land\\
And help me understand the best I can\\
And no one calls us to move on\\
And no one forces down our eyes\\
No one speaks\\
And no one tries\\
No one flies around the sun
\end{flushright}

Ker ima vsak surjektivna funkcija $f : A \to B$ prerez,\footnote{Pravimo, da je
  $g : B \to A$ \emph{prerez} funckije $f : A \to B$, če velja
  $f \circ g = \mathrm{id}_B$.} je svet lepši. Narobe pa bi bilo postaviti pripombo pred
ločilo\footnote{To je primer take pripombe.}, ali celo ob matematični simbol
$x$\footnote{Pozor, tu je mišljen $x$ in ne $x^3$.} in tako povzročati zmedo.

\section{Matematični izrazi}

O matematičnih izrazih bomo povedali še veliko, tu pa razložimo osnove. Matematične izraze
pišemo v enem od dveh \emph{matematičnih načinov}:
%
\begin{description}
\item \emph{vrstični način} (inline math), kjer je izraz zapisan med znakoma \texttt{\$}, na
  primer $a^2 + b^2 = c^2$. To je pravzaprav starejši zapis, lahko uporabimo tudi
  \texttt{{\char96}(} in \texttt{{\char96})}, na primer \(a^2 + b^2 = c^2\).

\item \emph{prikazni način} (display math), kjer je izraz zapisan med \texttt{{\char96}[} in
  \texttt{{\char96}]}, na primer
  %
  \[
    a^2 + b^2 = c^2.
  \]
  %
  Izraz v prikaznem načinu mora biti vedno del stavka in mora vsebovati ločilo, če je to
  smiselno. Starejši zapis za prikazni način je z \texttt{\$\$}:
  %
  $$
    a^2 + b^2 = c^2.
  $$
\end{description}

\noindent
%
Z ukazoma \texttt{{\char96}textstyle} in \texttt{{\char96}displaystyle} lahko nastavimo
vrstični ali prikazni način na roke, kar se denimo vidi, ko zapišemo ulomke. Ulomek
$\frac{2}{3}$ je v vrstičnem načinu, ulomek $\displaystyle \frac{2}{3}$ pa v prikaznem,
čeprav je znotraj vrstice. Kot vidimo, je to zelo grdo, saj so zaradi tega ulomka vrstice
med seboj preveč razmaknjene. Še najbolje bi bilo pisati $2/3$.

Matmatični izraz moramo vedno zapisati v celoti in ga ne delimo na več kosov. Tako je
pravilno $x + y = y + x$ in ne $x$ + $y$ = $y$ + $x$.

\emph{Vsak} matematični izraz mora biti v matematičnem načinu, če tudi je dolg samo en
znak. Nepravilno je zapisati ``naj bo x realno Å¡tevilo'', pravilno pa je ``naj bo $x$
realno Å¡tevilo''.


\section{Viri}

Navedimo nekaj virov za {\LaTeX}. Ker smo uporabili paket \texttt{hyperref}, lahko na povezave kar kliknete:
%
\begin{description}
\item[\href{http://www-lp.fmf.uni-lj.si/plestenjak/vaje/latex/lshort.pdf}{\emph{Ne najkrajši uvod v {\LaTeX}}}]
  Ravno pravšnji pregled {\LaTeX}a --- priporočamo!

\item[\url{https://www.sharelatex.com/learn/Main_Page}]
  Naučite se LaTeX v 30 minutah!

\item[\url{https://miktex.org}] LaTeX za MS Windows

\item[\url{http://www.tug.org/mactex/}] LaTeX za MacOS

\item[\url{https://www.tug.org/texlive/}] LaTeX za Linux (vendar ga namestita kar kot paket)

\item[\url{https://www.sharelatex.com}] LaTeX v vašem brskalniku, ne zahteva nobenega
  nameščanja opreme, poleg tega omogoča tudi hkratno urejanje večih avtorjev.

\item[\url{https://www.overleaf.com}]
  Alternativa za ShareLaTeX.

\item[\url{https://www.latex-project.org/help/documentation/}]
  Uradna dokumentacija za {\LaTeX}.

\item[\url{https://www.ctan.org}] Uradna zbirka vseh paketov za {\LaTeX}. Tu najdete
  dokumentacijo za pakete, ki pa jih ne nameščajte sami, ker jih lahko samodejno z vašim
  LaTeXom.

\item[\url{https://en.wikibooks.org/wiki/LaTeX}]
  Učbenik za LaTeX v angleščini.

\item[\url{http://tug.ctan.org/info/symbols/comprehensive/symbols-a4.pdf}]
  230 strani posebnih matematicnih znakov za {\LaTeX}.

\item[\url{http://detexify.kirelabs.org/classify.html}] Če iščete poseben znak, ga tu
  narišete in spletna stran vam pove, kateri ukaz v {\LaTeX}u vam da tak znak.
\end{description}

\end{document}
